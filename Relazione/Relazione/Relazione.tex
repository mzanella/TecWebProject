%ricordarsi di compilare più di una volta
\documentclass[a4paper,12pt]{article}

\usepackage{leaf}
\setcounter{table}{-1}

\titlepage{}

\author{Oscar Elia Conti, Federico Tavella, Marco Zanella}
\date{05/02/2016}
\intestazioni{Relazione Progetto Tecnologie Web}
\pagenumbering{gobble}
\begin{document}
	\begin{titlepage}
		\centering
		
		\vspace{1cm}
		{\huge\bfseries Gruppo FourMinusOne\par}
		\line(1,0){350} \\
		{\scshape\LARGE Università di Padova \par}
		\vspace{0.5cm}
		{\scshape\Large Relazione Progetto Tecnologie Web\par}
		\vspace{0.5cm}
		\includegraphics[scale=0.75]{immagini/logo.png}	 \\	
		\vspace{0.5cm}

\begin{tabular}{c|c}

{\hfill\textbf{Componenti}} & Oscar Elia Conti \parbox[t]{5cm}{} \\ & Federico Tavella \parbox[t]{5cm}{} \\ & Marco Zanella \parbox[t]{5cm}{} \\ \\

{\hfill\textbf{Referente}} & Oscar Elia Conti \parbox[t]{5cm}{} \\ & conti.oscarelia@gmail.com \\ \\


\end{tabular}
\vspace{0.5cm}


\textbf{Link al sito}
\url{http://tecnologie-web.studenti.math.unipd.it/tecweb/~oconti} \\
		\vspace{0.5cm}
		Credenziali utente\\
		\textbf{Username: } marco@email.com\\
		\textbf{Password: } password\\
		\vspace{0.5cm}
		Credenziali amministratore\\
		\textbf{Username: } oscar@email.com\\
		\textbf{Password: } password\\
		
		
		
\end{titlepage}

	\pagestyle{myfront}	
	
	\newpage
		\tableofcontents 	% produce l'indice delle sezioni del documento
	
	\label{LastFrontPage}

	\newpage
		\pagestyle{mymain}
	
	\newpage
		\subfile{sections/Abstract}
	
	\newpage
		\subfile{sections/Categorie_film}
	
	\newpage
		\subfile{sections/Utenti_destinatari}
		
	\newpage
		\subfile{sections/Gerarchia_file}
			
	\newpage
		\subfile{sections/Accessibilita}
		
	
	\newpage
		\subfile{sections/Struttura}
		
	\newpage
		\subfile{sections/Presentazione}
	
	\newpage
		\subfile{sections/Comportamento}
		
	\newpage
		\subfile{sections/Gestione_dati}
	
	\newpage
		\subfile{sections/Perl}
		
	\newpage
		\subfile{sections/Validazione}
		
	\newpage
		\subfile{sections/Organizzazione}
	
		
	\label{LastPage}

\end{document}
