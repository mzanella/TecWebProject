\documentclass[../Relazione.tex]{subfiles}

\begin{document}

\section{Validazione e Test}

\subsection{Validazione}

Per garantire la qualità del sito è stato usato il validatore messo a disposizione dal W3C per assicurasi che il codice HTML, CSS, XML, XSLT e Perl fosse valido.
Per poter validare i file .xsd presenti sulla cartella \url{http://tecnologie-web.studenti.math.unipd.it/tecweb/~oconti} è sufficiente aggiungere al tag radice di ogni file .xml la seguente stringa \textit{xmlns="http://tecnologie-web.studenti.math.unipd.it/tecweb/~oconti"}. Ogni singola pagina generata da uno script Perl è stata validata tramite il validatore W3C per le pagine HTML.

\subsection{Test}

Il sito è stato testato sui seguenti browser:
\begin{itemize}
\item Mozilla Firefox 44;
\item Google Chrome 48;
\item Microsoft Edge Windows 10;
25.10586.0.0
\item Internet Explorer 11.0;
\end{itemize}

e sui seguenti sistemi operativi

\begin{itemize}
\item Ubuntu (14.04 e 15.10);
\item Windows 10;
\end{itemize}

senza notare differenze rilevanti. Inoltre, si è effettuato un test disattivando JavaScript sui browser e le uniche differenze emerse sono:

\begin{itemize}
\item i suggerimenti interni ai campi delle form non vengono visualizzati;
\item gli errori nei campi delle form vengono segnalati solamente attraverso l'evento \textit{onsubmit};
\item il menù per la navigazione non scompare dopo il resize della finestra ma viene semplicemente modificata la sua presentazione.
\end{itemize}
			
\end{document}
