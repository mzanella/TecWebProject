\documentclass[../Relazione.tex]{subfiles}

\begin{document}

\section{Utenti destinatari}

Il sito si rivolge principalmente a due generi di utenti: chi usufruisce dei servizi e chi li fornisce. Il fornitore in questo caso è detto \textit{amministratore} ed ha il compito di inserire film nel database del sito (in una qualsiasi delle tre categorie), modificare e/o rimuovere film già caricati. L'utente che usufruisce del servizio invece ha la possibilità di accedere alle informazioni di un singolo film, ricercare un film per nome/categoria e, se ha effettuato l'accesso, di commentare un film. Riassumendo, le categorie di utenti sono:

\begin{itemize}
\item \textbf{Amministratore:} ha il compito di inserire film (in qualsiasi categoria), di rimuoverli e/o modificarli;
\item \textbf{Utente generico:} accede alle informazioni sui singoli film, a sua volta può essere:
\begin{itemize}
\item \textbf{Utente non registrato:} può semplicemente accedere ai singoli film ed effettuare una ricerca per nome/categoria;
\item \textbf{Utente registrato: } tutto ciò che può fare un utente non registrato e in aggiunta la possibilità di commentare i film (se ha effettuato l'accesso).
\end{itemize}
\end{itemize}
			
\end{document}
