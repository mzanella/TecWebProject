\documentclass[../Relazione.tex]{subfiles}

\begin{document}

\section{Categorie dei film}

Come indicato in precedenza, per i film sono state individuate 3 categorie. Le caratteristiche comuni ad ogni film, indipendentemente dalla categoria, sono:

\begin{itemize}

\item Nome del film;
\item Locandina del film;
\item Descrizione testuale della locandina;
\item Paese di produzione;
\item Durata;
\item Anno di produzione;
\item Data di uscita nelle sale;
\item Regia;
\item Attori;
\item Genere;
\item Trama.

\end{itemize}

Mentre per le singole categorie ci sono delle caratteristiche aggiuntive quali:

\begin{itemize}

\item \textbf{Film in uscita}: questi film non sono ancora presenti nelle sale del cinema ma è nota una data di uscita. Non possiedono caratteristiche aggiuntive rispetto a quelle descritte.

\item \textbf{Film ora in sala}: i film in questa categoria sono disponibili nelle sale cinematografiche. In aggiunta ai film in uscita è caratterizzato da:
\begin{itemize}
\item Incasso;
\item Recensione;
\item Valutazione (su una scala da 1 a 5).
\end{itemize}

\item \textbf{Film recensiti}: questi film non sono più presenti in sala ma sono stati recensiti, per cui o sono precedenti alla creazione del sito o sono stati rimossi dalla programmazione delle sale. Possiedono le stesse caratteristiche dei film ora in sala.

\end{itemize}

Questa suddivisione porta a tre categorie mutuamente esclusive tra di loro, perciò si può parlare di \textbf{schema organizzativo esatto} per quanto riguarda la suddivisione dei film.

\end{document}
